\chapter{Анализ \\ предметной области}
В данном разделе представлена информация об актуальности исследуемой задачи и определены основные понятия, используемые при дальнейшем анализе.
\section{Актуальность задачи}
В настоящее время социальные сети и форумы становятся частью повседневной жизни подавляющего количества пользователей сети Интернет. Активный обмен мнениями в сети привел к увеличению интереса к автоматическому извлечению тональности текста как со стороны научного сообщества, так и со стороны многих коммерческих организаций.

Например, крупнейшие интернет -- магазины, получают многочисленное количество отзывов на свои товары. Выделение общих закономерностей в большом количестве неструктурированных текстов --- это рутинная работа, требующая автоматизации. Таким образом, подбор эффективного алгоритма выделения является актуальной задачей экономической и производственной аналитики. 
%С ростом ложной информации в сети 

Однако, существует множество подходов для автоматического определения тональности текста, каждый из которых имеет свои особенности. В работе предлагается классификация этих методов и их краткий обзор.
\section{Основные понятия}
Очевидно, что анализ тональности текста тесно связан с понятием естественного языка. Определение естественного языка с точки зрения компьютерной лингвистики дано в работе \cite{ling-gen} следующим образом: \\
\textit{Естественный язык} --- большая открытая многоуровневая система знаков, возникшая для обмена информацией в процессе практической деятельности человека, и постоянно изменяющаяся в связи с этой деятельностью.

\textit{Тональностью текста} принято называть эмоциональную оценку, выраженную в неструктурированном тексте по отношению к сущности, и определяемую тональностью составляющих ее лексических единиц и правил их сочетания.

Автоматический анализ текста осуществляется на основе двух подходов: инженерно-лингвистический подход и подход на основе машинного обучения.

Инженерно лингвистический подход (подход с использованием паттернов и шаблонов) заключается в генерации правил, на основе которых будет определяться тональность текста. Для реализации таких методов создаются словари оценочной лексики и разрабатываются алгоритмы применения лингвистических правил для учета контекста употребляемых отдельно слов и выражений.

Подходы основанные на машинном обучении принято делить на классы, представленные ниже. 
\begin{itemize}
	\item Машинное обучение с учителем. Отбирается некоторое количество текстов для обучения и на их основе обучается классификатор.
	\item Машинное обучение без учителя. Подход основан на идее, что наибольший вес в тексте имеют самые часто встречаемые термины.\cite{ml-uns} Выделив данные термины, можно сделать вывод о тональности текста в целом.
\end{itemize}


Важным термином для работы с текстами на естественном языке является корпус текстов. Под корпусом текстов понимают проработанную под определенными правилами совокупность текстов, которая используется как база для исследования. Далее будет указано, что классификаторы, применяемые в методе анализа на основе машинного обучения, в основнм работают с корпусами текстов. 

\section{Некоторые сложности при анализе \\ тональности}

Тональность может выражаться с помощью эксплицитных и имплицитных оценок. Использование оценочной лексики --- это имплицитный способ выражения эмоциональной оценки. Например, высказывание \textit{<<фен хороший>>} --- это имплицитная оценка, поскольку высказывание содержит оценочное слово. Однако, высказывание \textit{<<расческа широкая и не для густых волос>>} не содержит оценочных слов, но в то же время дает эмоциональную оценку товару путем указания реального факта, приводящего к оценке.

Таким образом, имплицитное мнение (оценка)\cite{mining} --- это объективное высказывание, из которого следует оценка, т.\,е. имплицитное мнение сообщает желательный или нежелательный факт.

Оценочные слова в тексте часто сопровождаются словами --- модификаторами, которые либо меняют полярность слова, либо усиливают ее. Таким образом, при анализе тональности требуется иметь некоторую модель, которая учитывает модификацию тональности слова. Например, в работе \cite{modf} модификаторы полярности имеют процентную оценку <<усиления>>. Пример оценки некоторых модификаторов представлен в таблице \ref{modifiers}.
\captionsetup{singlelinecheck = false, justification=raggedleft}
\begin{table}[H]
	\caption{Процентная оценка <<усиления>> модификаторов}
	\renewcommand{\arraystretch}{1.2}
	\centering
	\begin{tabular}{ll}
		Модификатор & Усиление(\%) \\ \hline\hline
		& \\ 
		slightly & $-50$\\
		somewhat  & $-30$\\
		pretty  &$-10$\\
		really &$+15$\\
		very&$+25$\\
		extraordinarily&$+50$\\
		(the) most &$+100$\\ 
	\end{tabular}
	\label{modifiers}
\end{table}
Важной проблемой является определение сферы действия модификатора полярности в конкретном предложении. Например, \textit{<<Мне не нравится состав актеров в этом фильме, но некоторые актеры были очень запоминающимися.>>} Частица <<не>> модифицирует только слово <<нравится>>, но не модифицирует слово <<запоминающимися>>.
\section{Вывод}
Приведены некоторые определения, которые будут использованы в дальнейшем при обзоре методов анализа тональности текста. Сформулирована актуальность задачи.