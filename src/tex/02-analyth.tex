\chapter[Анализ предметной области]{Анализ \\ предметной области}
В данном разделе представлена информация об актуальности исследуемой задачи и определены основные понятия, используемые при дальнейшем анализе.

\section{Основные понятия}
Очевидно, что анализ тональности текста тесно связан с понятием естественного языка. Определение естественного языка с точки зрения компьютерной лингвистики дано в работе \cite{ling-gen} следующим образом: \\
\textit{Естественный язык} --- большая открытая многоуровневая система знаков, возникшая для обмена информацией в процессе практической деятельности человека, и постоянно изменяющаяся в связи с этой деятельностью.

\textit{Тональностью текста} принято называть эмоциональную оценку, выраженную в неструктурированном тексте по отношению к сущности, и определяемую тональностью составляющих ее лексических единиц и правил их сочетания.

Автоматический анализ текста осуществляется на основе двух подходов: инженерно-лингвистический подход и подход на основе векторного анализа.

Инженерно лингвистический подход (подход с использованием паттернов и шаблонов) заключается в генерации правил, на основе которых будет определяться тональность текста. Для реализации таких методов создаются словари оценочной лексики и разрабатываются алгоритмы применения лингвистических правил для учета контекста употребляемых отдельно слов и выражений.

Подходы основанные на векторном анализе принято делить на классы, представленные ниже. 
\begin{itemize}
	\item Машинное обучение с учителем. Отбирается некоторое количество текстов для обучения и на их основе обучается классификатор. 
	\item Машинное обучение без учителя. Подход основан на идее, что наибольший вес в тексте имеют самые часто встречаемые термины.\cite{ml-uns} Выделив данные термины, можно сделать вывод о тональности текста в целом.
\end{itemize}


Важным термином для работы с текстами на естественном языке является корпус текстов. Под корпусом текстов понимают проработанную под определенными правилами совокупность текстов, которая используется как база для исследования. Далее будет указано, что классификаторы, применяемые в методах на основе векторного анализа, в основнм работают с корпусами текстов. 

\section{Некоторые сложности при анализе \\ тональности}

Тональность может выражаться с помощью эксплицитных и имплицитных оценок. Использование оценочной лексики --- это имплицитный способ выражения эмоциональной оценки. Например, высказывание \textit{<<фен хороший>>} --- это имплицитная оценка, поскольку высказывание содержит оценочное слово. Однако, высказывание \textit{<<расческа широкая и не для густых волос>>} не содержит оценочных слов, но в то же время дает эмоциональную оценку товару путем указания реального факта, приводящего к оценке.

Таким образом, имплицитное мнение (оценка)\cite{mining} --- это объективное высказывание, из которого следует оценка, т.\,е. имплицитное мнение сообщает желательный или нежелательный факт.

Важной проблемой является определение сферы действия модификатора полярности в конкретном предложении. Например, \textit{<<Мне не нравится состав актеров в этом фильме, но некоторые актеры были очень запоминающимися.>>} Частица <<не>> модифицирует только слово <<нравится>>, но не модифицирует слово <<запоминающимися>>.
\section{Вывод}
Приведены некоторые определения, которые будут использованы в дальнейшем при обзоре методов анализа тональности текста. Сформулирована актуальность задачи.