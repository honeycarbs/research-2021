\chapter{Введение}
\pagenumbering{arabic}\setcounter{page}{2}
С развитием социальных сетей, сервисов онлайн --- покупок, рекомендательных сервисов о различных товарах и услугах растет актуальность автоматизации тональности текста. Автоматизация определения тональности текста --- это алгоритмически сложная задача, включающая некоторые не менее сложные подзадачи, о которых пойдет речь в представленной работе.

Анализ тональности текста часто рассматривается как часть научной области, называемой \textit{Text Mining} --- интеллектуальный анализ неструктурированных данных. Методы анализа неструктурированных данных лежат на стыке нескольких областей: обработка естественных языков, интеллектуальный анализ данных, поиск информации, извлечение информации и управление знаниями.\cite{data_mining}

Направление интеллектуального анализа текста, рассмотренное в данной работе --- задача анализа тональности текстов (Sentiment Analysis). Задача близка к классической задаче контент -- анализа текстов массовой коммуникации, в которой оценивается общая тональность высказываний в целом. 

Целью проводимой работы является классификация существующих методов анализа тональности текста, выделение прикладных задач, которые наиболее подходят для каждого класса.