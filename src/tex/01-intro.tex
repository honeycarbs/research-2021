\addchap{ВВЕДЕНИЕ}
\pagenumbering{arabic}\setcounter{page}{2}
В настоящее время социальные сети и форумы становятся частью повседневной жизни подавляющего количества пользователей сети Интернет. Активный обмен мнениями в сети привел к увеличению интереса к автоматическому извлечению тональности текста как со стороны научного сообщества, так и со стороны многих коммерческих организаций. Автоматизация определения тональности текста --- это алгоритмически сложная задача, включающая некоторые не менее сложные подзадачи, о которых пойдет речь в представленной работе.

Например, крупнейшие интернет-магазины, получают многочисленное количество отзывов на свои товары. Выделение общих закономерностей в большом количестве неструктурированных текстов --- это рутинная работа, требующая автоматизации. Таким образом, подбор эффективного алгоритма выделения является актуальной задачей экономической и производственной аналитики. 

Однако, существует множество подходов для автоматического определения тональности текста, каждый из которых имеет свои особенности. В работе предлагается классификация этих методов и их краткий обзор.

Анализ тональности текста часто рассматривается как часть научной области, называемой \textit{Text Mining} --- интеллектуальный анализ неструктурированных данных. Методы анализа неструктурированных данных лежат на стыке нескольких областей: обработка естественных языков, интеллектуальный анализ данных, поиск информации, извлечение информации и управление знаниями.\cite{data_mining}

Направление интеллектуального анализа текста, рассмотренное в данной работе --- задача анализа тональности текстов (Sentiment Analysis). Задача близка к классической задаче контент-анализа текстов массовой коммуникации, в которой оценивается общая тональность высказываний в целом. 

Целью проводимой работы является классификация существующих методов анализа тональности текста, выделение прикладных задач, которые наиболее подходят для каждого класса.