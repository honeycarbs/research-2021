\chapter[Определение тональности на основе правил и шаблонов]{Определение тональности \\ на основе правил \\и шаблонов}
В разделе описано применение инженерно --- лингвистического подхода, приведены примеры словарей оценочной лексики и примеры математических моделей, реализующих данный подход.
\section{Словари оценочной лексики}
При применении метода, основывающееся на правилах и шаблонах, сначала необходимо составить словарь оценочных слов и выражений. В таких словарях обычно каждому слову приписывается оценка тональности. Словари могут быть сформированы следующим образом:
\begin{itemize}
	\item сопоставление существующего словаря с корпусом анализируемых текстов, выборка предметной лексики;
	\item автоматизированный анализ корпуса текста и отбор оценочной лексики по заранее известному алгоритму;
	\item перевод иноязычных существующих словарей оценочной лексики на необходимый язык;
	\item просмотр корпуса текстов и пополнение словаря вручную.
\end{itemize}

В работе \cite{productrus} предложен подход автоматического порождения тонального словаря ProductSentiRus. Извлечение оценочных слов в заданной предметной области основано на
нескольких текстовых корпусах: корпус отзывов о продуктах с оценками пользователей, коллекции описаний продуктов и контрастного корпуса (например, новостного). Такие корпусы могут быть автоматически сформированы для разных предметных областей. 

Кроме того, в работе предполагается, что некоторые части корпуса мнений можно выделить: например, корпусы, в которых выше концентрация оценочных слов, большее разнообразие знаков пунктуации (троеточия и восклицательные знаки), большая концентрация коротких предложений (менее семи слов), предложения, содержащие аспект, к которому применена оценочная лексика, без других существительных.

Словарь представлен как список 5 тысяч слов, упорядоченных по мере снижения вычисленной вероятности их оценочности без указания позитивной или негативной тональности. Точность оценочных слов в первой тысяче слова списка составляет более 91\%. 


Ниже представлена таблица наиболее вероятных оценочных слов по версии словаря ProductSentiRus.
\begin{table}[H]
	\caption{Примеры слов из словаря ProductSentiRus}
	\centering
	\renewcommand{\arraystretch}{1.6}
	\begin{tabular}{ll}
		бесподобный & 0.963 \\
		невнятный & 0.953 \\
		отличнейший & 0.935 \\
		обалденный & 0.933 \\
		безумно & 0.924 \\
		непонятно & 0.921 \\
		неприятно & 0.920 \\
		отвратный & 0.920 \\
		нежный & 0.916
	\end{tabular}
	\label{psrus}
\end{table}

Также можно привести в пример словарь РуСентиЛекс. Он представляет собой упорядоченный по алфавиту список слов и выражений, содержащий следующие типы русскоязычных слов, значения которых связаны с тональностью: 
\begin{itemize}
	\item слова литературного русского языка, явно выражающие отношение к аспекту (например, \textit{хороший});
	\item слова литературного русского языка, подразумевающие в контексте наличие имплицитной оценки, то есть не передающие сами по себе отношение автора, но имеющие положительную или отрицательную оценку, например, \textit{болезнь, спам, драка};
	\item сленговые слова русскоязычных пользователей Твиттера.
\end{itemize}
Словарь РуСентиЛекс хранится в простом текстовом формате, подобном формату словаря MPQA \cite{mpqa}. Каждой единице словаря приписываются следующие атрибуты:
\begin{itemize}
	\item часть речи;
	\item слово или фраза, в которой каждое слово стоит в лемматизированной форме(для существительных --- именительный падеж, единственное число, для прилагательных --- именительный падеж, единственное число, мужской род, для глаголов, причастий, деепричастий --- глагол в инфинитиве (неопределённой форме) несовершенного вида);
	\item тональность (негативная, позитивная, нейтральная, двойная). Последнее означает, что оценка зависит от контекста;
	\item источник тональности (эксплицитная или имплицитная оценка).
\end{itemize}
Словаря Linis-Crowd \cite{linsucrowd} создавался для анализа тональности текстов социальных сетей. Солварь включает:
\begin{itemize}
	\item наиболее частотные прилагательные русского языка, употребляемые в текстах социальных сетей;
	\item образованные от отобранных прилагательных наречия,;
	\item словарь ProductSentiRus, из которого были выбраны слова, подходящие для анализа сообщений в социальных сетях и др.
\end{itemize}
Оценка выражается целым числом по шкале от $-$2 (сильно негативный) до +2 (сильно позитивный). Оценки различных разметчиков усреднялись.
\section{Построение математической модели}
Для правильности построения оценки предложения используется ряд правил, которые модифицируют оценку в зависимости от контекста. Список наиболее часто используемых правил приведен ниже. 
\begin{enumerate}
	\item Суммирование оценок слов, входящих в состав одного высказывания относящегося к определенной сущности.
	\item Применение правил отрицания. 
	\item Выставление негативной оценки для словосочетания, в котором употреблено хотя бы одно негативное выражение.\cite{mark}
\end{enumerate}
Применения правил отрицания требует более детального рассмотрения. Так, например, в работе \cite{negation} представлены правила, использующиеся при обнаружении слов <<не>>, <<никогда>>, и так далее. Правила можно сформулировать следующим образом:
\begin{itemize}
	\item отрицание отрицательной оценки $\rightarrow$ положительная оценка;
	\item отрицание положительной оценки $\rightarrow$ отрицательная оценка;
	\item отрицание слова без эмоциональной окраски $\rightarrow$ отрицательная оценка (например, <<не работает>>). 
\end{itemize}
С учетом применения правил отрицания определение тональности слова в данной работе имеет следующий вид (листинг \ref{nego-list}):

\begin{algorithm}[H]
	\caption{Определение тональности слова}\label{nego-list}
	\begin{algorithmic}
		\Procedure{оценкаСлова}{слово, предложение, оценка}
		\If{Слово выражает отрицание}
		\State{применить правила отрицания}
		\State{отметить слова в \textit{предложение}, к которым применено отрицание}
		\Else
		\State{оценить слово согласно \textit{оценка}}
		\EndIf
		\EndProcedure
	\end{algorithmic}
\end{algorithm}

Учитывая данное правило, проставление оценок происходит следующим образом. Пусть $ast_1 \dots sw_i$ --- оценочные термины, для которых уже проведена процедура оценки, $a_i$ --- оценка из словаря оценочной лексики. Тогда оценка для каждого аспектного термина вычисляется следующим образом: 
\begin{equation}
	score(a_i, S) = \sum_{sw_j \in S}\cfrac{num\left(sw_j\right)}{dist\left( sw_j, a_i\right)},
\end{equation}
где $num\left(sw_j\right)$ --- числовая оценка тональности, $dist\left( sw_j, a_i\right)$ --- расстояние между оценочным словом и аспектом. На оценку каждого оценочного термина влияют все слова в предложении, однако мера в которой они влияют на итоговую оценку определяется расстоянием от слова до оценочного термина. 

Работа \cite{modf} учитывает фактор нереального контекста --- в этой работе описываются правила, в результате применения которых тональность слов, являющихся индикаторами нереального контекста, не принимаются во внимание. В работе определены следующие маркеры нереального контекста: 
\begin{itemize}
	\item индикаторы условного наклонения (например, if);
	\item модальные глаголы;
	\item вопросы и слова, заключенные в кавычки.
\end{itemize}

Работа \cite{8pattrns} предлагает более детальный подход --- используется шесть правил для составления оценки:
\begin{itemize}
	\item перевод в противоположную тональность и применение отрицаний;
	\item приписывание доминирующей тональности идентификатора для синтаксических групп --- например, POS(\textit{завораживающий}) + NEG(\textit{хаос}) \\ = POS(\textit{завораживающий хаос});
	\item распространение модификатора на слово, стоящее рядом, если используется глагол распространения, такой как \textit{ненавидеть, обожать} или \textit{восхищаться};
	\item доминирование полярности глагола над объектом, к которому применяется глагол;
	\item нейтрализация оценочного выражения предлогом --- модификатором, таким как, например, \textit{не смотря на};
	\item усиление или ослабление веса тональности при обнаружении таких слов как \textit{очень}.
\end{itemize}

Таким образом, инженерно --- лингвистический подход имеет следующие достоинства:
\begin{itemize}
	\item результат работы зависит от содержимого словаря оценочной лексики и используемых правил --- то есть, результат предсказуем;
	\item возможен более глубокий анализ тональности на уровне высказывания, если подобрать словарь, оценка слов в котором будет шире шабл.
\end{itemize}
Однако, можно выделить и следующие недостатки:
\begin{itemize}
	\item составление словаря оценочной лексики и правил оценки вручную --- трудоемкая и дорогостоящая операция;
	\item при узком диапазоне слов в словаре оценочной лексики метод дает неточные результаты. 
\end{itemize}
\section{Вывод}
Инженерно лингвистический подход показывает высокую точность работы. Однако, результат напрямую зависит от организации и содержания словаря. Данный метод удобно применять в тех ситуациях, когда известна предметная область исследования и сущности, с которыми будет проведена работа.