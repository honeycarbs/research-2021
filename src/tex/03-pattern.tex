\chapter[Определение тональности на основе правил и шаблонов]{Определение тональности \\ на основе правил \\и шаблонов}
В разделе описано применение инженерно --- лингвистического подхода, приведены примеры словарей оценочной лексики и примеры математических моделей, реализующих данный подход.
\section{Словари оценочной лексики}
При применении метода, основывающееся на правилах и шаблонах, сначала необходимо составить словарь оценочных слов и выражений. В таких словарях обычно каждому слову приписывается оценка тональности. Словари могут быть сформированы следующим образом:
\begin{itemize}
	\item сопоставление существующего словаря с корпусом анализируемых текстов, выборка предметной лексики;
	\item автоматизированный анализ корпуса текста и отбор оценочной лексики по заранее известному алгоритму;
	\item перевод иноязычных существующих словарей оценочной лексики на необходимый язык;
	\item просмотр корпуса текстов и пополнение словаря вручную.
\end{itemize}

Например, словарь MPQA\cite{mpqa} содержит более 8000 отдельных слов, которым приписана оценка тональности и метка полярности(позитивный, нейтральный, негативный) и силой оценочного содержания(сильный или слабый).

В словаре ANEW дана более детальная оценка каждому слову. В отличие от словаря MPQA, словарь ANEW имеет три 9-балльные шкалы --- шкала удовольствия, шкала возбужденности и шкала контролируемости. Ниже представлены примеры некоторых слов из данного словаря (см. таблицу \ref{anewt}).

Большое внимание уделяется автоматическому порождению оценочных словарей. В работе \cite{auto} выделение оценочных прилагательных базируется на синтаксических шаблонах и союзах <<и>>, <<но>>, <<или>>. Предполагается, что оценочные термины, связанные этими союзами имеют одинаковую полярность, не считая случая собза <<но>>. 

Один из частых подходов извлечения словаря оценочных слов для заданной предметной области состоит в задании набора общезначимых оценочных слов, а затем пополнения этого набора на основе корпуса текстов. 

\captionsetup{singlelinecheck = false, justification=raggedleft}
\begin{table}[H]
	\caption{Примеры слов из словаря ANEW}
	\renewcommand{\arraystretch}{1.6}
	\begin{tabular}{llllll}
		Description & Word & Valence & Arousal & Dominance & Word \\
					& No.  & Mean(SD)& Mean(SD)& Mean (SD) & Frequency \\ \hline\hline
		abduction & 621 & 2.76 (2.06) & 5.53 (2.43)& 3.49 (2.38)&  1 \\
		abortion  & 622 & 3.50 (2.30) & 5.39 (2.80)& 4.59 (2.54)& 6 \\
		death 	  & 100 & 1.61 (1.40) & 4.59 (3.07)& 3.47 (2.50)& 277 \\
	\end{tabular}
	\label{anewt}
\end{table}

Анализируемый текст сопоставляется с имеющимся словарем. Тональность текста складывается из тональности предложений. Тональность предложения складывается из тональности присутствующих в нем слов, для которых прописана оценка.

\section{Построение математической модели}
Для правильности построения оценки предложения используется ряд правил, которые модифицируют оценку в зависимости от контекста. Список наиболее часто используемых правил приведен ниже. 
\begin{enumerate}
	\item Суммирование оценок слов, входящих в состав одного высказывания относящегося к определенной сущности.
	\item Применение правил отрицания. 
	\item Выставление негативной оценки для словосочетания, в котором употреблено хотя бы одно негативное выражение.\cite{mark}
\end{enumerate}
Применения правил отрицания требует более детального рассмотрения. Так, например, в работе \cite{negation} представлены правила, использующиеся при обнаружении слов <<не>>, <<никогда>>, и так далее. Правила можно сформулировать следующим образом:
\begin{itemize}
	\item отрицание отрицательной оценки $\rightarrow$ положительная оценка;
	\item отрицание положительной оценки $\rightarrow$ отрицательная оценка;
	\item отрицание слова без эмоциональной окраски $\rightarrow$ отрицательная оценка (например, <<не работает>>). 
\end{itemize}
С учетом применения правил отрицания определение тональности слова в данной работе имеет следующий вид (листинг \ref{nego-list}):

\begin{algorithm}[H]
	\caption{Определение тональности слова}\label{nego-list}
	\begin{algorithmic}
		\Procedure{оценкаСлова}{слово, предложение, оценка}
		\If{Слово выражает отрицание}
		\State{применить правила отрицания}
		\State{отметить слова в \textit{предложение}, к которым применено отрицание}
		\Else
		\State{оценить слово согласно \textit{оценка}}
		\EndIf
		\EndProcedure
	\end{algorithmic}
\end{algorithm}

Учитывая данное правило, проставление оценок происходит следующим образом. Пусть $ast_1 \dots sw_i$ --- оценочные термины, для которых уже проведена процедура оценки, $a_i$ --- оценка из словаря оценочной лексики. Тогда оценка для каждого аспектного термина вычисляется следующим образом: 
\begin{equation}
	score(a_i, S) = \sum_{sw_j \in S}\cfrac{num\left(sw_j\right)}{dist\left( sw_j, a_i\right)},
\end{equation}
где $num\left(sw_j\right)$ --- числовая оценка тональности, $dist\left( sw_j, a_i\right)$ --- расстояние между оценочным словом и аспектом. На оценку каждого оценочного термина влияют все слова в предложении, однако мера в которой они влияют на итоговую оценку определяется расстоянием от слова до оценочного термина. 

Работа \cite{modf} учитывает фактор нереального контекста --- в этой работе описываются правила, в результате применения которых тональность слов, являющихся индикаторами нереального контекста, не принимаются во внимание. В работе определены следующие маркеры нереального контекста: 
\begin{itemize}
	\item индикаторы условного наклонения (например, if);
	\item модальные глаголы;
	\item вопросы и слова, заключенные в кавычки.
\end{itemize}

Работа \cite{8pattrns} предлагает более детальный подход --- используется шесть правил для составления оценки:
\begin{itemize}
	\item перевод в противоположную тональность и применение отрицаний;
	\item приписывание доминирующей тональности идентификатора для синтаксических групп --- например, POS(\textit{beautiful}) + NEG(\textit{mess}) \\ = POS(\textit{beautiful mess});
	\item распространение модификатора на слово, стоящее рядом, если используется глагол распространения, такой как \textit{hate, adore} или \textit{admire};
	\item доминирование полярности глагола над объектом, к которому применяется глагол;
	\item нейтрализация оценочного выражения предлогом --- модификатором, таким как, например, \textit{despite};
	\item усиление или ослабление веса тональности при обнаружении таких слов как \textit{very, extremely}.
\end{itemize}

Таким образом, инженерно --- лингвистический подход имеет следующие достоинства:
\begin{itemize}
	\item результат работы зависит от содержимого словаря оценочной лексики и используемых правил --- то есть, результат предсказуем;
	\item возможен более глубокий анализ тональности на уровне высказывания, если подобрать словарь, оценка слов в котором будет шире, например, словарь ANEW.
\end{itemize}
Однако, можно выделить и следующие недостатки:
\begin{itemize}
	\item составление словаря оценочной лексики и правил оценки вручную --- трудоемкая и дорогостоящая операция;
	\item при узком диапазоне слов в словаре оценочной лексики метод дает неточные результаты. 
\end{itemize}
\section{Вывод}
Инженерно лингвистический подход показывает высокую точность работы. Однако, результат напрямую зависит от организации и содержания словаря. Данный метод удобно применять в тех ситуациях, когда известна предметная область исследования и сущности, с которыми будет проведена работа.