\addchap{ЗАКЛЮЧЕНИЕ}
На основе исследуемой литературы был проведен анализ методов определения тональности текста на естественном языке. Были выявлены две категории методов: методы на основе шаблонов и правил и методы на основе векторного анализа. 

В ходе работы были сделаны некоторые выводы, представленные ниже.
\begin{enumerate}
	\item Инженерно --- лингвистические методы требуют наличия словаря оценочной лексики. Чем обширнее словарь, тем точнее будет определена тональность текста.
	\item Методы векторного анализа требуют предварительно подготовленной обучающей коллекции для классификатора. Если подборка достаточно большая, то метод покажет наибольшую эффективность. 
	\item В отличие от методов на основе векторного анализа, в инженерно --- лингвистических методах требуется, чтобы заранее были известны сущности, с которыми работает алгоритм.  
	\item При использовании методов на основе векторного анализа трудно сменить предметную область для заранее построенной системы. 
\end{enumerate} 


Инженерно --- лингвистические методы предлагается использовать для тех случаев, когда заранее определены сущности оценивания: например, отзывы на определенную вещь в интернет --- магазинах. 

Методы на основе векторного анализа предлагается использовать для текстов, в которых чаще всего оцениваются сущности одной предметной области, но в разных ее аспектах. К таким текстам относятся личные блоги и новостные тексты. 